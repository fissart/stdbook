\documentclass[a4paper]{book}
%\usepackage[inline]{asymptote}
\usepackage{asymptote}
\usepackage{csvsimple}
\usepackage{numprint}
\usepackage{tabularx}
\usepackage{colortbl}
\usepackage{spreadtab}
\usepackage{expl3}
\ExplSyntaxOn
% make an internal function available to the user
\cs_set_eq:NN \fpeval \fp_eval:n
\ExplSyntaxOff
\def\asydir{asy}
%latexmk main.tex -ps -pdf- -dvi- -pvc
%live-server www.html in folder//npm install -g live-server ///
%while true; do inotifywait -e CLOSE_WRITE www.asy; asy -f html  www.asy; done; ///
%%%%%%%%%%%%%%%%%%%%%%%%%%%%%%%%%%%%%%%%

\usepackage[spanish]{babel}
\usepackage[centertags]{amsmath}
\usepackage{amsfonts}

\usepackage{graphicx}\usepackage{longtable}
\usepackage{booktabs}
%\usepackage{ulem}
%\usepackage{textcomp}
%\usepackage{showframe}
%\usepackage[utf8]{inputenc}
%\usepackage{hyperref}
%%%%%%%%%%%%%%%%%%%%%%%%%%%%%%%%%%%%%%%%
\usepackage[apaciteclassic, nosectionbib, tocbib]{apacite}
\usepackage{usebib}
\bibinput{bb}
\usepackage{makeidx}
\makeindex
%%%%%%%%%%%%%%%%%%%%%%%%%%%%%%%%%%%%%%%%
\newtheorem{comen}{Comentario}[chapter]
\newtheorem{thm}{Teorema}[chapter]
\newtheorem{defn}[thm]{Definición}
\newtheorem{lem}{Lema}[thm]
\newtheorem{cor}{Corolario}[thm]
\newtheorem{prop}{Proposicion}[thm]
\newtheorem{rem}{Remark}[thm]
\newtheorem{ill}{Illustration}[thm]
%%%%%%%%%%%%%%%%%%%%%%%%%%%%%%%%%%%%%%%%

\newcommand{\qw}{\phi}
\newcommand{\Real}{\mathbb R}
\newcommand{\pa}[1]{\left(#1\right)}
%%%%%%%%%%%%%%%%%%%%%%%%%%%%%%%%%%%%%%%%
%%%%%%%%%%%%%%%%%%%%%%%%%%%%%%%%%%%%%%%%
%%%%%%%%%%%%%%%%%%%%%%%%%%%%%%%%%%%%%%%%


\begin{document}
%%%%%%%%%%%%%%%%%%%%%%%%%%%%%%%%%%%%%%%%
\begin{asydef}
settings.prc=false;
defaultpen(fontsize(11 pt));
defaultpen(linewidth(0.7pt));
settings.render=2;
\end{asydef}
%%%%%%%%%%%%%%%%%%%%%%%%%%%%%%%%%%%%%%%%
\thispagestyle{empty}
{
\centering
\vspace{3cm}
\bf{\huge ARTE Y MATEMÁTICAS}\\
\bf{\large ARTE Y MATEMÁTICAS}\\
\vspace{0.5cm}
\bf{RICARDO}\\
\vspace{5cm}

\begin{asy}
import graph3;
size(300,0);
currentprojection=perspective(8,-8,8);
triple f(pair t) {
  real u=t.x;
  real v=t.y;
  real r=2-cos(u);
  real x=3*cos(u)*(1+sin(u))+r*cos(v)*(u < pi ? cos(u) : -1);
  real y=8*sin(u)+(u < pi ? r*sin(u)*cos(v) : 0);
  real z=r*sin(v);
  return (x,y,z);
}
surface s=surface(f,(0,0),(2pi,2pi),16,8,Spline);
draw(s,blue+opacity(0.7), meshpen=black+0.6bp,render(merge=true));

string lo="$\alpha=\sum_1^\infty$";
string hi="$\beta=\rho^3$";
real h=0.05;
begingroup3("parametrization");
//draw(surface(scale(0.08)*lo,s,0,1,h,bottom=false),"[0,0.5pi]");
//draw(surface(scale(0.08)*rotate(90)*hi,s,2,1,h,bottom=false),"[pi,2pi]");
endgroup3();
//axes3("$x$","$y$","$z$", Arrows3);
\end{asy}
\vfill
%ps://asy.marris.fr/
Departmento de mathemática y física, FIMGC USNCH\\
\emph{E-mail}: \texttt{ricardomallqui@gmail.com}\\
URL: \textsf{www.fractales.com}

}
\newpage
%%%%%%%%%%%%%%%%%%%%%%%%%%%%%%%%%%%%%%%%

 {
 	\thispagestyle{empty}
 	\noindent\bf{Arte y Matemáticas}\\
 	\bf{Ricardo Michel Mallqui Baños}\\
 	\vspace{3cm}

\noindent Un libro basado en codigo asymptote LaTeX y pstricks\\

\noindent Bibliografia.\\
\noindent Incluye Indice.\\
1. Geometry, Differential. 2. Curves. 3. Surfaces. \\
 	\vfill
 	\noindent
 		\begin{asy}
 	size(300);
 	path ltrans(path p,int d)
 	{
 	path a=rotate(65)*scale(0.4)*p;
 	return shift(point(p,(1/d)*length(p))-point(a,0))*a;
 	}
 	path rtrans(path p, int d)
 	{
 	path a=reflect(point(p,0),point(p,length(p)))*rotate(65)*scale(0.35)*p;
 	return shift(point(p,(1/d)*length(p))-point(a,0))*a;
 	}

 	void drawtree(int depth, path branch)
 	{
 	if(depth == 0) return;
 	real breakp=(1/depth)*length(branch);
 	draw(subpath(branch,0,breakp),blue);
 	drawtree(depth-1,subpath(branch,breakp,length(branch)));
 	drawtree(depth-1,ltrans(branch,depth));
 	drawtree(depth-1,rtrans(branch,depth));
 	return;
 	}

 	path start=(0,0)..controls (-1/10,1/3) and (-1/20,2/3)..(1/20,1);
 	drawtree(7,start);
 	\end{asy}

%\noindent \texttt{\textregistered\;\textcopyright\; 2023 pa-esfa, Inc. UNSCH, Huamanga}\\
\noindent %\texttt{\textregistered\;\textcopyright}\\
\texttt{Todos los derechos reservados. Ninguna parte de esto
libro puede ser reproducido en cualquier forma,
o por cualquier medio, sin permiso
por escrito del editor.}\\
Departmento de mathemática y física, FIMGC USNCH\\
\emph{E-mail}: \texttt{ricardomallqui@gmail.com}\\
URL: \textsf{www.fractales.com}

 }
%%%%%%%%%%%%%%%%%%%%%%%%%%%%%%%%%%%%%%%%
\newpage
\renewcommand\listfigurename{Índice general}
\pagenumbering{roman}
\setcounter{page}{1}
\addcontentsline{toc}{chapter}{Índice general}
\tableofcontents
\renewcommand\listfigurename{Lista de figuras}
\addcontentsline{toc}{chapter}{Lista de figuras}
\listoffigures

\renewcommand\listtablename{Lista de tablas}
\addcontentsline{toc}{chapter}{Lista de tablas}
\listoftables
\newpage

\clearpage
%%%%%%%%%%%%%%%%%%%%%%%%%%%%%%%%%%%%%%%%

\chapter*{Presentación}
\addcontentsline{toc}{chapter}{Presentación}

\underline{\underline{Double underlined text}}
{Double underlined text}
\textsl{\underline{Slanted underlined}}
\textsc{\underline{Small caps underlined}}












%%%%%%%%%%%%%%%%%%%%%%%%%%%%%%%%%%%%%%%%
%%%%%%%%%%%%%%%%%%%%%%%%%%%%%%%%%%%%%%%%
%%%%%%%%%%%%%%%%%%%%%%%%%%%%%%%%%%%%%%%%
%%%%%%%%%%%%%%%%%%%%%%%%%%%%%%%%%%%%%%%%
%%%%%%%%%%%%%%%%%%%%%%%%%%%%%%%%%%%%%%%%
\chapter{Curvas}
\pagenumbering{arabic}
\setcounter{page}{1}

\begin{table}[!ht]
\caption{123456789}
\centering
\begin{asy}
import math;
size(300,0);

int[] x={2,3,5,7,11,14,3};

int y=-1;

for (int k : x){
if(k==7){
label("$\frac{1}{2}$"+string(k),(-3,--y));
}else
{
label("$\int^{"+string(k)+"}_{"+string(k)+"}$",(-2,--y),E,blue);
}

}
int l=0;
for (int k=0; k<15;++k){

label("$\displaystyle\int^{"+string(k)+"}_{"+string(k)+"}f(x)$",(-8,-l-2.1*k),E,blue);

}

label(scale(1.1)*"$\int_{e}^{3}\epsilon$",(-2,0.5));


void tablepythagore(int m=9, int n=9){
	int[][] y=new int[m+1][n+1];

	for (int i=1; i<=m; ++i)
	{

			for (int j=1; j<=n; ++j)
			{
			y[i][j]=i*j;
			if(i!=2&&j!=1){
			label(scale(.9)*format("%i",y[i][j]),(j,-i));
		    }
			if(i==1){
			label(format("%i",j),(j,0),paleblue);
			//dot((j,0));
			}
			}

		label(format("%i",i),(0,-i),red);
		label(format("%i",i-1),(1,-i),red);
	}
	draw((-2,-0.5)--(-2,-9.5), orange+linewidth(0.5mm));
draw((-0.5,-0.5)--(9.5,-0.5), orange+linewidth(0.5mm));
draw((-0.5,-9.5)--(9.5,-9.5), orange+linewidth(0.5mm));
//add(shift(-.5,-.5-m)*grid(n+1,m+1,orange));
label(scale(1.1)*"$\int_{e}^{3}\epsilon$",(-0.5,0.5));
}

tablepythagore();

shipout(bbox(.25cm,Fill(white)));
\end{asy}

\end{table}
When we put (vertically) large expressions inside of parentheses (or brackets, or curly braces, etc.), the parentheses don't resize to fit the expression and instead remain relatively small. For instance, $$f(x) = \pi(\frac{\sqrt{x}}{x-1})$$ comes out as

When we put (vertically) large expressions inside of parentheses (or brackets, or curly braces, etc.), the parentheses don't resize to fit the expression and instead remain relatively small. For instance, $$f(x) = \pi(\frac{\sqrt{x}}{x-1})$$ comes out as



\begin{figure}[!ht]
\centering	\begin{asy}
import graph;
size(12cm,7cm,IgnoreAspect);
typedef real realfcn(real);
realfcn F(real p) {
return new real(real t) {return 1/(sqrt(2*pi)*(1/(4*p)))*exp(-(t-1/(2*p))^2/(2*(1/(4*p))^2));};
}
for(int i=1; i < 7; ++i){
real rho=(1/(4*i));
real mu=(1/(2*i));
draw(graph(F(i),-0.1,1, n=200, Hermite),Pen(i),
"$\frac{1}{"+format( "%.3f", rho)+"\sqrt{2\pi}}{e^{-\frac{(x-"+format( "%.3f", mu)+")^2}{2("+format( "%.3f", rho)+")^2}}}$");
}
label("$\displaystyle\frac{1}{\rho\sqrt{2\pi}}{e^{-\frac{(x-\mu)^2}{2(\rho)^2}}}$",(0.5,8));
xaxis("$x$",0.1,LeftTicks);
yaxis("$y$",0,LeftTicks);
//xaxis(BottomTop,LeftTicks);
//yaxis(LeftRight,RightTicks(trailingzero));
//yaxis("$y$",LeftRight,RightTicks(trailingzero));
//attach(legend(),truepoint(E),20E,UnFill);
attach(legend(2),(point(S).x,truepoint(S).y),1S);
\end{asy}
\caption{wwwwww}
\end{figure}





\begin{figure}[!ht]
\centering	\begin{asy}
import graph;
size(12cm,7cm,IgnoreAspect);
typedef real realfcn(real);
realfcn F(real p) {
	return new real(real t) {return 1/(pi*(p)*(1+((t-5)/(p))^2));};
}
for(int i=1; i < 7; ++i){
real rho=(1/(4*i));
real mu=(1/(2*i));
draw(graph(F(i),0,10, n=200, Hermite),Pen(i),
"$\frac{1}{p\pi\left(1+\frac{t-5}{p^2}\right)}$");
}
label("$\displaystyle\frac{1}{p\pi\left(1+\frac{t-5}{p^2}\right)}$",(7,0.3));
xaxis("$x$",0.1,LeftTicks);
yaxis("$y$",0,LeftTicks);
//xaxis(BottomTop,LeftTicks);
//yaxis(LeftRight,RightTicks(trailingzero));
//yaxis("$y$",LeftRight,RightTicks(trailingzero));
//attach(legend(),truepoint(E),20E,UnFill);
attach(legend(2),(point(S).x,truepoint(S).y),1S);
\end{asy}
\caption{wwwwww}
\end{figure}







When we put (vertically) large expressions inside of parentheses (or brackets, or curly braces, etc.), the parentheses don't resize to fit the expression and instead remain relatively small. For instance, $$f(x) = \pi(\frac{\sqrt{x}}{x-1})$$ comes out as

When we put (vertically) large expressions inside of parentheses (or brackets, or curly braces, etc.), the parentheses don't resize to fit the expression and instead remain relatively small. For instance, $$f(x) = \pi(\frac{\sqrt{x}}{x-1})$$ comes out as

When we put (vertically) large expressions inside of parentheses (or brackets, or curly braces, etc.), the parentheses don't resize to fit the expression and instead remain relatively small. For instance, $$f(x) = \pi(\frac{\sqrt{x}}{x-1})$$ comes out as





\begin{figure}[!ht]
	\centering
	\begin{asy}
	import graph;
	size(300,0);
	int a=-1, b=2;
	real f(real x) {return x^3-x+2;}
	real g(real x) {return x^2;}
	draw(graph(f,a,b,operator ..),red);
	draw(graph(g,a,b,operator ..),blue);
	xaxis();
	int n=30;
	real width=(b-a)/(real) n;
	path w=graph(f,a,b,operator ..);
	path ww=graph(g,a,b,operator ..);
	path h=buildcycle((a,g(a))--(a,f(a)),w,(b,f(b))--(b,g(b)),ww);
	fill(h,orange);
	draw(h,black+linewidth(0.3mm));
	labelx("$a$",a);
	labelx("$b$",b);
	draw((a,0)--(a,g(a)),dotted);
	draw((b,0)--(b,g(b)),dotted);
	real m=a+0.73*(b-a);
	arrow("$f(x)$",(m,f(m)),N,red);
	arrow("$g(x)$",(m,g(m)),E,0.8cm,blue);
	int j=2;
	real xi=b-j*width;
	real xp=xi+width;
	real xm=0.5*(xi+xp);
	pair dot=(xm,0.5*(f(xm)+g(xm)));
	dot(dot,green+4.0);
	arrow("$\left(x,\frac{f(x)+g(x)}{2}\right)$",dot,NE,2cm,green);
	\end{asy}
	\caption{$f(x)$ wwww $\left(x,\frac{f(x)+g(x)}{2}\right)$ www}
\end{figure}


When we put (vertically) large expressions inside of parentheses (or brackets, or curly braces, etc.), the parentheses don't resize to fit the expression and instead remain relatively small. For instance, $$f(x) = \pi(\frac{\sqrt{x}}{x-1})$$ comes out as

When we put (vertically) large expressions inside of parentheses (or brackets, or curly braces, etc.), the parentheses don't resize to fit the expression and instead remain relatively small. For instance, $$f(x) = \pi(\frac{\sqrt{x}}{x-1})$$ comes out as

When we put (vertically) large expressions inside of parentheses (or brackets, or curly braces, etc.), the parentheses don't resize to fit the expression and instead remain relatively small. For instance, $$f(x) = \pi(\frac{\sqrt{x}}{x-1})$$ comes out as


\begin{longtable}{>{\color{blue}}ccc>{\color{blue}}c>{\color{yellow}}c}
	\caption{Combinaciones de los tres segmentos de la seccion aurea.}
	\label{tab:w1wwwww}\\
	\toprule
	\textbf{Ecuación} & \textbf{Simplicación} & \textbf{Raices}& \multicolumn{2}{c}{\textbf{Raices simplicación}}\\\midrule
	 &  &  & $x_1$ & $x_2$ \\
	\midrule
	\endfirsthead % <-- This denotes the end of the header, which will be shown on the first page only
 \multicolumn{4}{c}{{\bfseries \tablename\ \thetable{} -- continua de la página anterior}} \\
	\toprule
	\textbf{Value 1} & \textbf{Value 2} & \textbf{Value 3}& \multicolumn{2}{c}{\textbf{Raices}}\\\midrule
	$\alpha$ & $\beta$ & $\gamma$ & $x_1$ \\
	\midrule
	\endhead
	\multicolumn{4}{c}{{Continúa en la proxima página}} \\ \midrule
	\endfoot
	\bottomrule
	\endlastfoot
	$\frac{x}{x-1}=\frac{x-1}{1}$&$ x^2-3x+1=0 $ & $x=\frac{3\pm\sqrt{5}}{2}$  & $x_1=\fpeval{round((1+sqrt(5))/2,3)}$ & $x_2=\fpeval{round((1-sqrt(5))/2,3)}$\\\midrule
	$\frac{x+1}{x}=\frac{x}{1}$&$ x^2-x-1=0$     & $x=\frac{1\pm\sqrt{5}}{2}$  & $x_1=\fpeval{round((-1+sqrt(5))/2,3)}$ & $x_2=\fpeval{round((1-sqrt(5))/2,3)}$\\\midrule
	$\frac{x+1}{x}=\frac{x}{1}$&$ x^2-x-1=0$     & $x=\frac{1\pm\sqrt{5}}{2}$  & $x_1=\fpeval{round((3+sqrt(5))/2,3)}$ & $x_2=\fpeval{round((1-sqrt(5))/2,3)}$\\\midrule
	$\frac{1}{x}  =\frac{x}{1-x}$&$ x^2+x-1=0 $  &  $x=\frac{-1\pm\sqrt{5}}{2}$& $x_1=\fpeval{round((-3+sqrt(5))/2,3)}$ & $x_2=\fpeval{round((1-sqrt(5))/2,3)}$\\\midrule
	$\frac{x+1}{x}=\frac{x}{1}$&$ x^2-x-1=0$     & $x=\frac{1\pm\sqrt{5}}{2}$  & $x_1=\fpeval{round((1+sqrt(5))/2,3)}$ & $x_2=\fpeval{round((1-sqrt(5))/2,3)}$\\\midrule
	$\frac{x+1}{x}=\frac{x}{1}$&$ x^2-x-1=0$     & $x=\frac{1\pm\sqrt{5}}{2}$  & $x_1=\fpeval{round((1+sqrt(5))/2,3)}$ & $x_2=\fpeval{round((1-sqrt(5))/2,3)}$\\

\end{longtable}


\begin{longtable}{>{\color{blue}}ccc>{\color{blue}}c>{\color{yellow}}cccccccccc}
	\caption{Combinaciones de los tres segmentos de la seccion aurea.}
	\label{tab:w1wwwww}\\
	\toprule
$Y_i$	&	$f_i$	&	$F_i$	&	$F_i*$	&	$h_i$	&	$H_i$	&	$H_i$	&	$h_i\%$	&	$H_i\%$	&	$H_i*\%$	\\
	\midrule

	\endfirsthead
 \multicolumn{8}{c}{{\bfseries \tablename\ \thetable{} -- continua de la página anterior}}\\
	\toprule
$Y_i$	&	$f_i$	&	$F_i$	&	$F_i*$	&	$h_i$	&	$H_i$	&	$H_i$	&	$h_i\%$	&	$H_i\%$	&	$H_i*\%$	\\
	\endhead
	\midrule
	\multicolumn{8}{c}{{Continúa en la proxima página}} \\ \midrule
	\endfoot
	\bottomrule
	\endlastfoot

2	&	1	&	1.0000	&	20.0000	&	0.0500	&	0.0500	&	0.0025	&	0.2500	&	0.2500	&	0.2500	\\
3	&	2	&	3.0000	&	19.0000	&	0.1000	&	0.1500	&	0.0050	&	0.5000	&	0.5000	&	0.7500	\\
4	&	5	&	8.0000	&	17.0000	&	0.2500	&	0.4000	&	0.0125	&	1.2500	&	1.2500	&	2.0000	\\
5	&	7	&	15.0000	&	12.0000	&	0.3500	&	0.7500	&	0.0175	&	1.7500	&	1.7500	&	3.7500	\\
6	&	4	&	19.0000	&	5.0000	&	0.2000	&	0.9500	&	0.0100	&	1.0000	&	1.0000	&	4.7500	\\
7	&	1	&	20.0000	&	1.0000	&	0.0500	&	1.0000	&	0.0025	&	0.2500	&	0.2500	&	5.0000	\\
	&	20	&		&		&		&		&		&	5	&		&		\\

\end{longtable}




\begin{longtable}{>{\color{blue}}ccc>{\color{blue}}c>{\color{yellow}}cccccccccc}
	\caption{Combinaciones de los tres segmentos de la seccion aurea.}
	\label{tab:w1wwwww}\\
	\toprule
$Y_i$	&	$f_i$	&	$F_i$	&	$F_i*$	&	$h_i$	&	$H_i$	&	$H_i$	&	$h_i\%$	&	$H_i\%$	&	$H_i*\%$	\\
	\midrule

	\endfirsthead
 \multicolumn{8}{c}{{\bfseries \tablename\ \thetable{} -- continua de la página anterior}}\\
	\toprule
$Y_i$	&	$f_i$	&	$F_i$	&	$F_i*$	&	$h_i$	&	$H_i$	&	$H_i$	&	$h_i\%$	&	$H_i\%$	&	$H_i*\%$	\\
	\endhead
	\midrule
	\multicolumn{8}{c}{{Continúa en la proxima página}} \\ \midrule
	\endfoot
	\bottomrule
	\endlastfoot

2	&	1	&	1.0000	&	20.0000	&	0.0500	&	0.0500	&	0.0025	&	0.2500	&	0.2500	&	0.2500	\\
3	&	2	&	3.0000	&	19.0000	&	0.1000	&	0.1500	&	0.0050	&	0.5000	&	0.5000	&	0.7500	\\
4	&	5	&	8.0000	&	17.0000	&	0.2500	&	0.4000	&	0.0125	&	1.2500	&	1.2500	&	2.0000	\\
5	&	7	&	15.0000	&	12.0000	&	0.3500	&	0.7500	&	0.0175	&	1.7500	&	1.7500	&	3.7500	\\
6	&	4	&	19.0000	&	5.0000	&	0.2000	&	0.9500	&	0.0100	&	1.0000	&	1.0000	&	4.7500	\\
7	&	1	&	20.0000	&	1.0000	&	0.0500	&	1.0000	&	0.0025	&	0.2500	&	0.2500	&	5.0000	\\
	&	20	&		&		&		&		&		&	5	&		&		\\

\end{longtable}


\begin{longtable}{>{\color{blue}}ccc>{\color{blue}}c>{\color{yellow}}cccccccccc}
	\caption{Combinaciones de los tres segmentos de la seccion aurea.}
	\label{tab:w1wwwww}\\
	\toprule
$Y_i$	&	$f_i$	&	$F_i$	&	$F_i*$	&	$h_i$	&	$H_i$	&	$H_i$	&	$h_i\%$	&	$H_i\%$	&	$H_i*\%$	\\
	\midrule

	\endfirsthead
 \multicolumn{8}{c}{{\bfseries \tablename\ \thetable{} -- continua de la página anterior}}\\
	\toprule
$Y_i$	&	$f_i$	&	$F_i$	&	$F_i*$	&	$h_i$	&	$H_i$	&	$H_i$	&	$h_i\%$	&	$H_i\%$	&	$H_i*\%$	\\
	\endhead
	\midrule
	\multicolumn{8}{c}{{Continúa en la proxima página}} \\ \midrule
	\endfoot
	\bottomrule
	\endlastfoot

2	&	1	&	1.0000	&	20.0000	&	0.0500	&	0.0500	&	0.0025	&	0.2500	&	0.2500	&	0.2500	\\
3	&	2	&	3.0000	&	19.0000	&	0.1000	&	0.1500	&	0.0050	&	0.5000	&	0.5000	&	0.7500	\\
4	&	5	&	8.0000	&	17.0000	&	0.2500	&	0.4000	&	0.0125	&	1.2500	&	1.2500	&	2.0000	\\
5	&	7	&	15.0000	&	12.0000	&	0.3500	&	0.7500	&	0.0175	&	1.7500	&	1.7500	&	3.7500	\\
6	&	4	&	19.0000	&	5.0000	&	0.2000	&	0.9500	&	0.0100	&	1.0000	&	1.0000	&	4.7500	\\
7	&	1	&	20.0000	&	1.0000	&	0.0500	&	1.0000	&	0.0025	&	0.2500	&	0.2500	&	5.0000	\\
	&	20	&		&		&		&		&		&	5	&		&		\\

\end{longtable}
\begin{longtable}{>{\color{blue}}ccc>{\color{blue}}c>{\color{yellow}}cccccccccc}
	\caption{Combinaciones de los tres segmentos de la seccion aurea.}
	\label{tab:w1wwwww}\\
	\toprule
$Y_i$	&	$f_i$	&	$F_i$	&	$F_i*$	&	$h_i$	&	$H_i$	&	$H_i$	&	$h_i\%$	&	$H_i\%$	&	$H_i*\%$	\\
	\midrule

	\endfirsthead
 \multicolumn{8}{c}{{\bfseries \tablename\ \thetable{} -- continua de la página anterior}}\\
	\toprule
$Y_i$	&	$f_i$	&	$F_i$	&	$F_i*$	&	$h_i$	&	$H_i$	&	$H_i$	&	$h_i\%$	&	$H_i\%$	&	$H_i*\%$	\\
	\endhead
	\midrule
	\multicolumn{8}{c}{{Continúa en la proxima página}} \\ \midrule
	\endfoot
	\bottomrule
	\endlastfoot

2	&	1	&	1.0000	&	20.0000	&	0.0500	&	0.0500	&	0.0025	&	0.2500	&	0.2500	&	0.2500	\\
3	&	2	&	3.0000	&	19.0000	&	0.1000	&	0.1500	&	0.0050	&	0.5000	&	0.5000	&	0.7500	\\
4	&	5	&	8.0000	&	17.0000	&	0.2500	&	0.4000	&	0.0125	&	1.2500	&	1.2500	&	2.0000	\\
5	&	7	&	15.0000	&	12.0000	&	0.3500	&	0.7500	&	0.0175	&	1.7500	&	1.7500	&	3.7500	\\
6	&	4	&	19.0000	&	5.0000	&	0.2000	&	0.9500	&	0.0100	&	1.0000	&	1.0000	&	4.7500	\\
7	&	1	&	20.0000	&	1.0000	&	0.0500	&	1.0000	&	0.0025	&	0.2500	&	0.2500	&	5.0000	\\
	&	20	&		&		&		&		&		&	5	&		&		\\

\end{longtable}
\begin{longtable}{>{\color{blue}}ccc>{\color{blue}}c>{\color{yellow}}cccccccccc}
	\caption{Combinaciones de los tres segmentos de la seccion aurea.}
	\label{tab:w1wwwww}\\
	\toprule
$Y_i$	&	$f_i$	&	$F_i$	&	$F_i*$	&	$h_i$	&	$H_i$	&	$H_i$	&	$h_i\%$	&	$H_i\%$	&	$H_i*\%$	\\
	\midrule

	\endfirsthead
 \multicolumn{8}{c}{{\bfseries \tablename\ \thetable{} -- continua de la página anterior}}\\
	\toprule
$Y_i$	&	$f_i$	&	$F_i$	&	$F_i*$	&	$h_i$	&	$H_i$	&	$H_i$	&	$h_i\%$	&	$H_i\%$	&	$H_i*\%$	\\
	\endhead
	\midrule
	\multicolumn{8}{c}{{Continúa en la proxima página}} \\ \midrule
	\endfoot
	\bottomrule
	\endlastfoot

2	&	1	&	1.0000	&	20.0000	&	0.0500	&	0.0500	&	0.0025	&	0.2500	&	0.2500	&	0.2500	\\
3	&	2	&	3.0000	&	19.0000	&	0.1000	&	0.1500	&	0.0050	&	0.5000	&	0.5000	&	0.7500	\\
4	&	5	&	8.0000	&	17.0000	&	0.2500	&	0.4000	&	0.0125	&	1.2500	&	1.2500	&	2.0000	\\
5	&	7	&	15.0000	&	12.0000	&	0.3500	&	0.7500	&	0.0175	&	1.7500	&	1.7500	&	3.7500	\\
6	&	4	&	19.0000	&	5.0000	&	0.2000	&	0.9500	&	0.0100	&	1.0000	&	1.0000	&	4.7500	\\
7	&	1	&	20.0000	&	1.0000	&	0.0500	&	1.0000	&	0.0025	&	0.2500	&	0.2500	&	5.0000	\\
	&	20	&		&		&		&		&		&	5	&		&		\\

\end{longtable}
\begin{longtable}{>{\color{blue}}ccc>{\color{blue}}c>{\color{yellow}}cccccccccc}
	\caption{Combinaciones de los tres segmentos de la seccion aurea.}
	\label{tab:w1wwwww}\\
	\toprule
$Y_i$	&	$f_i$	&	$F_i$	&	$F_i*$	&	$h_i$	&	$H_i$	&	$H_i$	&	$h_i\%$	&	$H_i\%$	&	$H_i*\%$	\\
	\midrule

	\endfirsthead
 \multicolumn{8}{c}{{\bfseries \tablename\ \thetable{} -- continua de la página anterior}}\\
	\toprule
$Y_i$	&	$f_i$	&	$F_i$	&	$F_i*$	&	$h_i$	&	$H_i$	&	$H_i$	&	$h_i\%$	&	$H_i\%$	&	$H_i*\%$	\\
	\endhead
	\midrule
	\multicolumn{8}{c}{{Continúa en la proxima página}} \\ \midrule
	\endfoot
	\bottomrule
	\endlastfoot

2	&	1	&	1.0000	&	20.0000	&	0.0500	&	0.0500	&	0.0025	&	0.2500	&	0.2500	&	0.2500	\\
3	&	2	&	3.0000	&	19.0000	&	0.1000	&	0.1500	&	0.0050	&	0.5000	&	0.5000	&	0.7500	\\
4	&	5	&	8.0000	&	17.0000	&	0.2500	&	0.4000	&	0.0125	&	1.2500	&	1.2500	&	2.0000	\\
5	&	7	&	15.0000	&	12.0000	&	0.3500	&	0.7500	&	0.0175	&	1.7500	&	1.7500	&	3.7500	\\
6	&	4	&	19.0000	&	5.0000	&	0.2000	&	0.9500	&	0.0100	&	1.0000	&	1.0000	&	4.7500	\\
7	&	1	&	20.0000	&	1.0000	&	0.0500	&	1.0000	&	0.0025	&	0.2500	&	0.2500	&	5.0000	\\
	&	20	&		&		&		&		&		&	5	&		&		\\

\end{longtable}





\begin{figure}[!ht]
	\centering
	\begin{asy}
	import graph;
	size(300,0);
	int a=0, b=2;
	real f(real x) {return 1/(sqrt(2*pi)*(0.5))*exp(-(x-1)^2/(2*(0.5)^2));}
	real g(real x) {return 0;}
	path w=graph(f,a,b,operator ..);
	draw(graph(f,a-1,b+1,operator ..),orange+linewidth(0.3mm));
	//draw(graph(g,a,b,operator ..),black);
	xaxis();
	int n=50;
	path h=(a,0)--w--(b,0)--cycle;
	fill(h,orange);
	draw(h,black+linewidth(0.3mm));
	labelx("$a$",a);
	labelx("$b$",b);
	pair mid=(a+0.5*(b-a),(f(a+0.5*(b-a))+g(a+0.5*(b-a)))/2);
	label("$90\%$",mid,white);
	real m=a+0.5*(b-a);
	real p=a-0.1;
	real q=b+0.1;
	//arrow("$f(x)$",(m,f(m)),N,red);
	arrow("$5\%$",(p,0.5*f(p)),NW,orange);
	dot((p,0.5*f(p)),orange);
	arrow("$5\%$",(q,0.5*f(q)),NE,orange);
	dot((q,0.5*f(q)),orange);
	//arrow("$g(x)$",(m,g(m)),dir(-90),0.8cm,blue);
	\end{asy}
	\caption{www}
\end{figure}


\begin{figure}[!ht]
	\centering
	\begin{asy}
	import stats;
import graph;
size(12cm,6cm,false);
real[] tabxi={0,5,8,10,12,15,20,30,35};
real[] tabni={5,6,7,5,6,2,1,0};
real[] tabhi;
for(int i=0; i < tabni.length; ++i)
  tabhi[i]=tabni[i]/(tabxi[i+1]-tabxi[i]);

histogram(tabxi,tabhi,low=0,bars=true,yellow);
xaxis("wwwwwwww",Bottom, RightTicks(Step=2,step=1),above=true);
shipout(bbox(3mm,white));
	\end{asy}
	\caption{www}
\end{figure}



\begin{figure}[!ht]
	\centering
	\begin{asy}
	size(7.5cm,0);
// Tableau des modalités
string[] tabmod={"Modalit\'e 1","Modalit\'e 2","Modalit\'e 3",
                 "Modalit\'e 4","Modalit\'e 5"};
// Tableau des effectifs (ou fréquences)
real[] tabeff={20,6,7,10,11};
// Tableau des décalages éventuels des secteurs
real[] tabdecsec={0,.1,0,.2,0};
// Tableau des décalages éventuels des labels
real[] tabdeclab={0,.5,.5,.2,.2};
// Les deux couleurs utilisées pour composer
// les couleurs des secteurs
pen color1=green,color2=blue;
// Le stylo pour les labels
pen p3=blue,p4=yellow+white;
///////////////////////////////////////////////////////////////
/// Ce qui suit n'est a priori pas à changer et pourrait être
/// ajouté un de ces jours à une extension perso stats_gm.asy
/// pour être remplacé par :
/// diacirculaire(tabmod,tabeff,tabdec,color1,color2);
///////////////////////////////////////////////////////////////
real[] tabangle,tabanglecumule,tabanglelabel;
tabanglecumule[0]=0;
int n=tabeff.length;
for(int i=0; i<n; ++i) {
  tabangle[i]=tabeff[i]*360/sum(tabeff);
  tabanglecumule[i+1]=tabanglecumule[i]+tabangle[i];
  tabanglelabel[i]=tabanglecumule[i]+tabangle[i]/2;
  path secteur=(0,0)--arc((0,0),1,tabanglecumule[i],tabanglecumule[i+1])--cycle;
  transform t1=shift(tabdecsec[i]*dir(tabanglelabel[i])),
            t2=shift((.5+tabdecsec[i]+tabdeclab[i])*dir(tabanglelabel[i]));
  filldraw(t1*secteur,i/n*color1+(1-i/n)*color2+white);
  label(tabmod[i],t2*(0,0),p3,Fill(p4));
}
shipout(bbox(3mm,white));
	\end{asy}
	\caption{www}
\end{figure}






















\begin{figure}[!ht]
	\centering
	\begin{asy}
	size(7.5cm,0);

void bargraph(real X, real Y,
              real ymin, real ymax, real ystep,
              real tickwidth, string yformat,
              Label LX, Label LY, Label[] LLX,
              real[] height,
              pen p=nullpen){
    draw((0,0)--(0,Y),EndArrow);
    draw((0,0)--(X,0),EndArrow);
    label(LX,(X,0),plain.SE,fontsize(9));
    label(LY,(0,Y),plain.N,fontsize(9));
    real yscale=Y/(ymax+ystep);
    for(real y=ymin; y<ymax; y+=ystep) {
        draw((-tickwidth,yscale*y)--(0,yscale*y));
        label(format(yformat,y),(-tickwidth,yscale*y),plain.W,fontsize(9));
    }
    int n=LLX.length;
    real xscale=X/(2*n+2);
    for(int i=0;i<n;++i) {
        real x=xscale*(2*i+1);
        path P=(x,0)--(x,height[i]*yscale)--(x+xscale,height[i]*yscale)--(x+xscale,0)--cycle;
        fill(P,p);
        draw(P);
        label(LLX[i],(x+xscale/2),plain.S,fontsize(10));
    }
    for(int i=0;i<n;++i)
        draw((0,height[i]*yscale)--(X,height[i]*yscale),dashed);
}

string yf="%#.1f";
Label[] LX={"Printemps","Et\'e","Automne","Hiver"};
for(int i=0;i<LX.length;++i) LX[i]=rotate(45)*LX[i];
real[] H={12.9,21.3,9.8,4.3};

bargraph(X=60,Y=100,
         ymin=2,ymax=24,ystep=2,
         tickwidth=1,
         yf,
         "Saison","$\theta$ moyen",
         LX,H,
         yellow);
	\end{asy}
	\caption{www}
\end{figure}



\begin{figure}[!ht]
	\centering
	\begin{asy}
/* Illustration aidant au calcul de la médiane
   d'une série à caractère quantitatif continu
   d'après le polygone des effectifs cumulés croissants
   par interpolation linéaire */

/* Variables à modifier */

// Définition de la taille de l'image
size(7cm,5cm,false);
// Points A et B définissant l'un des segments du polygone
pair A=(8,37),B=(10,57.5);
// Ordonnée du point de [AB] dont on cherche l'abscisse
real yM=50;

/* A priori, ce qui suit ne doit pas être modifié
   la figure va s'adapter aux valeurs données précédemment */

real xM=(yM-A.y)*(B.x-A.x)/(B.y-A.y)+A.x;
real dx=.2(B.x-A.x), dy=.2(B.y-A.y);
draw((A.x-dx,A.y-2dy)--(B.x+dx,A.y-2dy),.7bp+black);
draw((A.x-2dx,A.y-dy)--(A.x-2dx,B.y+dy),.7bp+black);

draw(A--B,1.2bp+black);
dot("$A$",A,SW,blue);
dot("$B$",B,NE,blue);
dot("$M$",(xM,yM),SE,red);

draw((A.x-2dx,A.y)--A--(A.x,A.y-2dy),dashed+.5bp+black);
draw((A.x-2dx,B.y)--B--(B.x,A.y-2dy),dashed+.5bp+black);
draw((A.x-2dx,yM)--(xM,yM)--(xM,A.y-2dy),dashed+.5bp+black);
label(format("$%f$",A.x),(A.x,A.y-2dy),S);
label(format("$%f$",A.y),(A.x-2dx,A.y),W);
label(format("$%f$",B.x),(B.x,A.y-2dy),S);
label(format("$%f$",B.y),(A.x-2dx,B.y),W);
label("$x_M$?",(xM,A.y-2dy),S,red);
label(format("$%f$",yM),(A.x-2dx,yM),W);
	\end{asy}
	\caption{www}
\end{figure}















\bibliographystyle{apacite}
\bibliography{bb}
\addcontentsline{toc}{chapter}{Indices}
\printindex



\appendix
\pagenumbering{roman}
\setcounter{page}{1}
\chapter{Sistemas de coordenadas}
wwwwwwwwwwwwwwwwwwwwwwwwwwwwwwwwww
 wwwwwwwwwwwwwwwwwwwwwwwwwwwwwww
\end{document}
/*
real[][] f(real a, real b, real c)
{
    if (b^2-4*a*c<0) {
      real[][] w={
      {-b/2,(sqrt(abs(b^2-4*a*c)))/2*a,-b/2,(-b+sqrt(abs(b^2-4*a*c)))/2*a}//,{0,0}
      };
      return w;
   }
     real[][] w={
     {(-b+sqrt(abs(b^2-4*a*c)))/2*a,(-b-sqrt(abs(b^2-4*a*c)))/2*a}//,{0,0}
     };
     return w;
}

write(f(1,4,9));
write(f(1,4,9)[0][3]);
*/
